\documentclass[12pt]{jarticle}

\usepackage[dvipdfmx]{graphicx}
\usepackage{okumacro}
\usepackage[dvipdfmx]{pict2e}
\usepackage{url}
%% \renewcommand{\kanjifamilydefault}{\gtdefault} % ゴシック体にする
\renewcommand{\rmdefault}{lmss}

\pagestyle{empty}

% 名刺マルチカード エーワン株式会社 品番「51865」にあわせた設定
% 参照: http://www.a-one.co.jp/product/search/detail.php?id=51865
\setlength{\hoffset}{-1truein}  % 左マージン外を0に
\setlength{\voffset}{-1truein}  % 上マージン外を0に
\addtolength{\hoffset}{-4truemm} % 実測値を元に調整
\setlength{\oddsidemargin}{14truemm} % 名刺マルチカードの左マージンにあわせる
\setlength{\topmargin}{11truemm} % 名刺マルチカードの上マージンにあわせる
\setlength{\textwidth}{182truemm} % 名刺マルチカードの幅
\setlength{\textheight}{275truemm} % 名刺マルチカードの高さ
\setlength{\headheight}{0truemm}
\setlength{\headsep}{0truemm}
\setlength{\marginparsep}{0truemm}
\setlength{\marginparwidth}{0truemm}
\setlength{\footskip}{0truemm}
\setlength{\marginparpush}{0truemm}

\begin{document}
\setlength{\unitlength}{1truemm}
\begin{picture}(182,275)(0,0)
  %% \put(0,0){\dashbox{1.0}(182, 275){}} % 描画領域全体の枠線
  \multiput(0,0)(0,55){5}{%
    \multiput(0,0)(91,0){2}{%
      %% ここから中身 %%%%%%%%%%%%
      \begin{picture}(91,55)(0,0)
        %% \put(0,0){\dashbox{1.0}(91, 55){}} % 片ごとの枠線
        %% \put(5,10){\dashbox{4.0}(80, 40){}} % テキスト位置の枠線
        \put(5,40){\large \ruby{織}{お}\ruby{田}{だ} \ruby{信}{のぶ}\ruby{長}{なが}} %氏名
        \put(30,40){\small Nobunaga Oda} %name

        \put(5,30){\small Email:}
        \put(20,30){\small {\texttt nobunaga.oda@example.com}} % e-mail
        \put(5,25){\small URL:}
        \put(20,25){\small \url{http://www.example.com/}} % URL
        \put(5,20){\small TEL:}
        \put(20,20){\small 000-0000-0000} % TEL
        \put(5,15){\small 住所:}
        \put(20,15){\footnotesize 〒521-1311 滋賀県近江八幡市安土町下豊浦} % 住所

        \put(5,10){\scriptsize \dag この名刺は\LaTeX で作成しました。}
      \end{picture}
  }}
\end{picture}
\end{document}
